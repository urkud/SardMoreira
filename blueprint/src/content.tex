% In this file you should put the actual content of the blueprint.
% It will be used both by the web and the print version.
% It should *not* include the \begin{document}
%
% If you want to split the blueprint content into several files then
% the current file can be a simple sequence of \input. Otherwise It
% can start with a \section or \chapter for instance.

\chapter{Preliminaries}%
\label{cha:preliminaries}

\section{Measure Theory}%
\label{sec:measure-theory}

Most of the lemmas in this section
are true for measures on general measurable spaces
with appropriate typeclass assumptions.
\texttt{Mathlib} versions of these lemmas should be stated
in the right generality.
However, for this project, it's acceptable to prove the lemmas
for Haar measures on finite dimensional real normed spaces only.

\begin{lemma}%
  \label{lem:measurable-meas-ball}
  \lean{Measurable.measure_apply}
  \leanok%
  Let \(E\) and \(F\) be finite dimensional real normed spaces.
  Let \(s \subset E \times F\) be a measurable set in their product.
  Let \(\mu\) be an additive Haar measure on \(E\).
  Then \(\varphi(x, y, r) = \mu \{z \in E\mid \|z - x\| < r \wedge (z, y) \in s\}\)
  is a measurable function in \((x, y, r)\).
\end{lemma}
\begin{proof}
  \leanok%
  Consider the set \(t \subset E \times E \times F \times \mathbb R\)
  given by \(t = \{(z, x, y, r) \mid \|z - x\| < r \wedge (z, y) \in s\}\).
  Clearly, this is a measurable set.
  Thus the measure of its sections \(t_{x, y, r}=\{z \mid (z, x, y, r) \in t\}\)
  is a measurable function of \((x, y, r)\),
  which is exactly what we need to prove.
\end{proof}

\begin{corollary}%
  \label{cor:measurable-density}
  \lean{Besicovitch.ae_tendsto_measure_sectl_inter_closedBall_div}
  \leanok%
  Let \(E\), \(F\), \(s\), and \(\mu\) be as in \autoref{lem:measurable-meas-ball}.
  Then the set of points \((x, y)\) such that \(x\) is a Lebesgue density point of \(s_{y} = \{z \mid (z, y) \in s\}\)
  is a measurable set.
\end{corollary}
\begin{proof}
  \uses{lem:measurable-meas-ball}
  \leanok%
  The theorem immediately follows from the non-parametrized version
  and measurability of the set of fiberwise density points of a measurable set.
\end{proof}

\section{\(C^k\) maps}%
\label{sec:ck-maps}

\section{\(C^{k}\) maps with locally Hölder derivatives}

In this section,
\begin{itemize}
\item \(k\) is a natural number;
\item \(\alpha\) is a real number between \(0\) and \(1\);
\item \(E\), \(F\), and \(G\) a real normed spaces;
\item \(f\colon E \to F\) is a function.
\end{itemize}

\begin{definition}%
  \label{def:cdh-at}
  We say that a map \(f\colon E\to F\) is \emph{\(C^{k+(\alpha)}\)} at a point \(a\),
  if it is \(C^{k}\) at \(a\) and \(f^{(k)}(x) - f^{(k)}(a) = O(\|x - a\|^{\alpha})\) as \(x\to a\).
\end{definition}

Note that this notion is weaker than a more commonly used \(C^{k+\alpha}\) smoothness assumption.
\begin{lemma}%
  \label{def:contdiffholder-imp-cdh-at}
  \uses{def:cdh-at}
  If \(f\colon E \to F\) is \(C^{k+\alpha}\) on an open set \(U\),
  i.e., \(f\) is \(C^{k}\) on \(U\) and \(f^{(k)}\) is \(\alpha\)-Hölder on \(U\),
  then \(f\) is \(C^{k+(\alpha)}\) at every point \(a \in U\).
\end{lemma}
\begin{proof}
  The proof follows immediately from definitions.
\end{proof}

\begin{lemma}%
  \label{lem:cdh-at-zero}
  \uses{def:cdh-at}
  A map \(f\colon E\to F\) is \(C^{k+(0)}\) at \(a\) iff it is \(C^{k}\) at \(a\).
\end{lemma}

\begin{proof}
  The forward implication follows from the definition.
  For the backward implication,
  we need to show that \(f^{(k)}(x) - f^{(k)}(a)=O(1)\) as \(x\to a\),
  which immediately follows from the continuity of \(f^{(k)}\) at \(a\).
\end{proof}

\begin{lemma}%
  \label{lem:cdh-at-mono}
  \uses{def:cdh-at}
  Let \(f\colon E\to F\) be a map which is \(C^{k+(\alpha)}\) at \(a\).
  Let \(l\) be a natural number and \(\beta \in [0, 1]\) be a real number.
  If \((l, \beta) \le (k, \alpha)\) in the lexicographic order,
  then \(f\) is \(C^{l+(\beta)}\) at \(a\).
\end{lemma}

\begin{proof}
  Note that \(l\le k\), hence \(f\) is \(C^{l}\) at \(a\).
  In order to show \(f^{(l)}(x) - f^{(l)}(a) = O(\|x - a\|^{\beta})\) as \(x\to a\),
  consider the cases \(l = k\), \(\beta \le \alpha\) and \(l < k\).

  In the former case, we have \(f^{(k)}(x) - f^{(k)}(a) = O\left(\|x - a\|^{\alpha}\right)=O\left(\|x - a\|^{\beta}\right)\) as \(x\to a\).

  In the latter case, \(f^{(l)}\) is differentiable at \(a\), hence \(f^{(l)}(x)-f^{(l)}(a)=O(x - a)=O(\|x - a\|^{\beta})\),
  where the latter estimate holds since \(\beta \le 1\).
\end{proof}

\begin{lemma}%
  \label{lem:cdh-at-comp}
  \uses{def:cdh-at}
  Consider \(g\colon F \to G\), \(f\colon E \to F\), and \(a \in E\)
  such that \(g\) is \(C^{k+(\alpha)}\) at \(f(a)\) and \(f\) is \(C^{k+(\alpha)}\) at \(a\), where \(k > 0\).
  Then \(g\circ f\) is \(C^{k+(\alpha)}\) at \(a\).
\end{lemma}

\begin{proof}
  Since \(g\) is \(C^{k}\) at \(f(a)\) and \(f\) is \(C^{k}\) at \(a\),
  the composition \(g\circ f\) is \(C^{k}\) at \(a\).
  Let us show that \({(g\circ f)}^{(k)}(x) - {(g\circ f)}^{(k)}(a) = O(\|x - a\|^{\alpha})\) as \(x\to \alpha\).

  Due to the Faa Di Bruno formula, we have
  \[
    {(g\circ f)}^{(k)}(a) = \sum_{P}TC_{P}\left(g^{(|P|)}(f(a)), f^{(|P_{0}|)}(a), \dots, f^{\left(\left|P_{|P|-1}\right|\right)}(a)\right),
  \]
  where the sum is taken over all partitions \(P\) of \(0, \dots, k - 1\) into disjoint sets,
  and \(TC_{P}\) is a continuous multilinear map in all \(|P|+1\) variables
  which depends only on the normed spaces and the partition \(P\).

  Subtracting these expressions, we see that it suffices to show that for each partition \(P\),
  we have
  \begin{align*}
    TC_{P}&\left(g^{(|P|)}(f(x)), f^{(|P_{0}|)}(x), \dots, f^{\left(\left|P_{|P|-1}\right|\right)}(x)\right) -\\
          &TC_{P}\left(g^{(|P|)}(f(a)), f^{(|P_{0}|)}(a), \dots, f^{\left(\left|P_{|P|-1}\right|\right)}(a)\right) =\\
          &O(\|x - a\|^{\alpha}).
  \end{align*}
  Since \(k > 0\), the function \(f\) is Lipschitz near \(a\),
  thus \(f(x) - f(a) = O(x - a)\), \(g^{(|P|)}(f(x)) - g^{(|P|)}(f(a)) = O(\|x - a\|^{\alpha})\), and \(f^{(|P_{j})}(x) - f^{(|P_{j})}(a) = O(\|x - a\|^{\alpha})\).
  Now, the goal follows from the fact that a continuous multilinear map is locally Lipschitz.
\end{proof}

\begin{definition}%
  \label{def:cdh-near}
  \uses{def:cdh-at}
  We say that a map \(f\colon E\to F\) is \emph{\(C^{k+(\alpha)}\)} near a set \(s\),
  if it is \(C^{k+(\alpha)}\)-smooth at all points of \(s\).
  Note that this definition depends on the values of \(f\) near \(s\), not only on \(s\).
\end{definition}

\begin{lemma}%
  \label{lem:cdh-on-nhds-iff-open}
  \uses{def:cdh-near}
  A map \(f\colon E\to F\) is \(C^{k+(\alpha)}\) near \(s\)
  if and only if there exits an open superset \(U\supset s\)
  such that \(f\) is \(C^{k}\) on \(U\)
  and for any \(a \in s\) we have \(f^{(k)}(x) - f^{(k)}(a) = o(\|x - a\|^{\alpha})\) as \(x\to a\).
\end{lemma}

\begin{proof}
  The reverse implication is trivial, we just take \(U\) as the neighborhood of each \(a \in s\) required in the definition.
  For the forward implication, choose an open neighborhood \(U_{a}\) of each \(a \in s\) such that \(f\) is \(C^{k}\) on \(U_{a}\).
  Then \(f\) is \(C^{k}\) on \(U = \bigcup_{a \in s} U_{a}\).
\end{proof}

\begin{remark}
  In the original paper, the definition of a \(C^{k+(\alpha)}\) map
  involves an open set \(U\) and its subset \(K\subset U\).
  However, we use a definition that does not fix \(U\) instead
  and prove equivalence of two definitions in \autoref{lem:cdh-on-nhds-iff-open}.
\end{remark}

\section{Vitali Families}%
\label{sec:vitali-families}

\section{Inverse function theorem}%
\label{sec:inverse-funct-theor}


In this section we prove a version of the inverse function theorem for \(C^{k+(\alpha)}\) functions,
then deduce some corollaries.
\chapter{Local charts}%
\label{cha:local-charts}

\section{Functions whose zeros include a given set}

\begin{lemma}%
  \label{lem:chart-one-deriv-zero}
  \uses{def:cdh-near}
  Consider a function \(f\colon E\to F\) between real normed spaces.
  Let \(k\) be a positive natural number, let \(\alpha\) be a real number in \([0, 1]\).
  Let \(s\) be a set such that
  \begin{itemize}
  \item \(f\) is \(C^{k+(\alpha)}\) near \(s\) in the sense of \autoref{def:cdh-near};
  \item \(df = 0\) at all points of \(s\).
  \end{itemize}
  Let \(a \in s\) be a Lebesgue 
\end{lemma}


%%% Local Variables:
%%% mode: latex
%%% TeX-master: "print.tex"
%%% End:
