% In this file you should put the actual content of the blueprint.
% It will be used both by the web and the print version.
% It should *not* include the \begin{document}
%
% If you want to split the blueprint content into several files then
% the current file can be a simple sequence of \input. Otherwise It
% can start with a \section or \chapter for instance.

\chapter{Introduction}%
\label{cha:introduction}

The goal of this project is to formalize Moreira's version of Sard's Theorem,
see \url{http://dmle.icmat.es/pdf/PUBLICACIONSMATEMATIQUES_2001_45_01_06.pdf}.

In this chapter, we describe some differences
between the original paper and the formalization,
as well as some minor inaccuracies found during formalization.

\chapter{Preliminaries}%
\label{cha:preliminaries}

\section{Measure Theory}%
\label{sec:measure-theory}

Most of the lemmas in this section
are true for measures on general measurable spaces
with appropriate typeclass assumptions.
\texttt{Mathlib} versions of these lemmas should be stated
in the right generality.
However, for this project, it's acceptable to prove the lemmas
for Haar measures on finite dimensional real normed spaces only.

\begin{lemma}%
  \label{lem:measurable-meas-ball}
  \lean{Measurable.measure_apply}
  \leanok%
  Let \(E\) and \(F\) be finite dimensional real normed spaces.
  Let \(s \subset E \times F\) be a measurable set in their product.
  Let \(\mu\) be an additive Haar measure on \(E\).
  Then \(\varphi(x, y, r) = \mu \{z \in E\mid \|z - x\| < r \wedge (z, y) \in s\}\)
  is a measurable function in \((x, y, r)\).
\end{lemma}
\begin{proof}
  \leanok%
  Consider the set \(t \subset E \times E \times F \times \mathbb R\)
  given by \(t = \{(z, x, y, r) \mid \|z - x\| < r \wedge (z, y) \in s\}\).
  Clearly, this is a measurable set.
  Thus the measure of its sections \(t_{x, y, r}=\{z \mid (z, x, y, r) \in t\}\)
  is a measurable function of \((x, y, r)\),
  which is exactly what we need to prove.
\end{proof}

\begin{corollary}%
  \label{cor:measurable-density}
  \lean{Besicovitch.ae_tendsto_measure_sectl_inter_closedBall_div}
  \leanok%
  Let \(E\), \(F\), \(s\), and \(\mu\) be as in \autoref{lem:measurable-meas-ball}.
  Then the set of points \((x, y)\) such that \(x\) is a Lebesgue density point of \(s_{y} = \{z \mid (z, y) \in s\}\)
  is a measurable set.
\end{corollary}
\begin{proof}
  \uses{lem:measurable-meas-ball}
  \leanok%
  The theorem immediately follows from the non-parametrized version
  and measurability of the set of fiberwise density points of a measurable set.
\end{proof}

\begin{lemma}%
  \label{lem:meas-ball-lowersemicont}
  Let \(X\) be a (pseudo) metric space with a measure \(\mu\).
  Consider the measure of an open ball \(\mu(B_{r}(x))\) as a function on \(X\times\mathbb R\),
  where \(\mathbb R\) is equipped with the upper topology.
  Then this function is lower semicontinuous.
\end{lemma}

\begin{proof}
  Consider a point \((x, r)\) and an extended nonnegative real number \(m < \mu(B_{r}(x))\)
  Since the union of the balls \(B_{r'}(x)\) over \(r' < r\) is the ball \(B_{r}(x)\),
  there exists \(r' < r\) such that \(\mu\left(B_{r'}(x)\right) > m\).
  Take \(r'' \in (r', r)\), then \(\mu\left(B_{\rho}(y)\right) \ge \mu\left(B_{r''}(x)\right)> m\)
  whenever \(y\in B_{r' - r''}(x)\) and \(\rho > r'\).
\end{proof}

\begin{corollary}%
  \label{cor:meas-ball-lowersemicont}
  Let \(X\) be a (pseudo) metric space with a measure \(\mu\).
  Then \(\mu(B_{r}(x))\) is lower semicontinuous in \((x, r)\in X\times\mathbb R\).
\end{corollary}

\begin{proof}
  \uses{lem:meas-ball-lowersemicont}
  This lemma immediately follows from \autoref{lem:meas-ball-lowersemicont}
  and the fact that the upper topology is coarser than the order topology.
\end{proof}

\begin{corollary}%
  \label{cor:meas-ball-mesaurable}
  The measure of an open ball in a (pseudo) metric space is measurable in \((x, r)\in X\times\mathbb R\).
\end{corollary}

\begin{proof}
  \uses{cor:meas-ball-lowersemicont}
  This statement immediately follows from \autoref{cor:meas-ball-lowersemicont}
  and the fact that a lower semicontinuous function is measurable.
\end{proof}

\begin{corollary}%
  \label{cor:meas-ball-min}
  If \(X\) is a (pseudo) metric space with a measure \(\mu\)
  and \(s\) is a nonempty compact set in \(X\),
  then for each \(r\) there exists \(x \in s\) such that \(\mu(B_{r}(x)) \le \mu(B_{r}(y))\) for all \(y \in s\).
\end{corollary}

\begin{proof}
  \uses{cor:meas-ball-lowersemicont}
  This statement immediately follows from \autoref{cor:meas-ball-lowersemicont}
  and the fact that a lower semicontinuous function on a nonempty compact
  achieves its minimal value.
\end{proof}

\begin{corollary}%
  \label{cor:meas-ball-gt-pos}
  If \(X\) is a compact (pseudo) metric space
  and \(\mu\) is a measure on \(X\) that is positive on nonempty open sets,
  then for each \(r > 0\) there exists \(\epsilon>0\)
  such that \(\mu(B_{r}(x))>\epsilon\) for all \(x\).
\end{corollary}

\begin{proof}
  \uses{cor:meas-ball-min}
  If \(X\) is nonempty, then we choose \(x\) as in \autoref{cor:meas-ball-min},
  then choose \(0 < \epsilon < \mu(B_{r}(x))\).
  Otherewise, we choose any \(\epsilon > 0\) (e.g., \(\epsilon = 1\)).
\end{proof}

\todo{Add similar statements about closed balls and upper semicontinuity.}

\section{Vitali Families}%
\label{sec:vitali-families}

TODO:\@ generalize some statement about Vitali families to outer measures.

\section{\(C^k\) maps}%
\label{sec:ck-maps}

\begin{lemma}%
  \label{lem:Ck-differentiable-iteratedFDeriv}
  \lean{ContDiffAt.differentiableAt_iteratedFDeriv}
  \leanok%
  If \(f \colon E \to F\) is \(C^{k}\) at \(a\) and \(l < k\),
  then \(D^{l}f\) is differentiable at \(a\).
\end{lemma}
\begin{proof}
  \leanok%
  Since \(f\) is \(C^{k}\) at \(a\) and \(l + 1 \le k\),
  \(D^{l}f\) is \(C^{1}\) at \(a\),
  thus it's differentiable at \(a\).
\end{proof}

\begin{lemma}%
  \label{lem:iteratedFDeriv-prod}
  If \(f\colon E \to F\) and \(g\colon E \to G\) are \(C^{k}\) at \(a\),
  then \(D^{k}(x \mapsto (f(x), g(x)))(a) = (D^{k}f(a), D^{k}g(a))\).
\end{lemma}
\begin{proof}
  This lemma immediately follows from uniqueness of the iterated derivative
  and lemmas in the library.
\end{proof}

\section{\(C^{k}\) maps with locally Hölder derivatives at a point}%
\label{sec:cdh-at}

In this section,
\begin{itemize}
\item \(k\) is a natural number;
\item \(\alpha\) is a real number between \(0\) and \(1\);
\item \(E\), \(F\), and \(G\) a real normed spaces;
\item \(f\colon E \to F\) is a function.
\end{itemize}

\begin{definition}%
  \label{def:cdh-at}
  \lean{ContDiffHolderAt}
  \leanok%
  We say that a map \(f\colon E\to F\) is \emph{\(C^{k+(\alpha)}\)} at a point \(a\),
  if it is \(C^{k}\) at \(a\) and \(D^{k}f(x) - D^{k}f(a) = O(\|x - a\|^{\alpha})\) as \(x\to a\).
\end{definition}

Note that this notion is weaker than a more commonly used \(C^{k+\alpha}\) smoothness assumption.
\begin{lemma}%
  \label{def:contdiffholder-imp-cdh-at}
  \lean{ContDiffHolderAt.of_contDiffOn_holderWith}
  \leanok%
  \uses{def:cdh-at}
  If \(f\colon E \to F\) is \(C^{k+\alpha}\) on an open set \(U\),
  i.e., \(f\) is \(C^{k}\) on \(U\) and \(D^{k}f\) is \(\alpha\)-Hölder on \(U\),
  then \(f\) is \(C^{k+(\alpha)}\) at every point \(a \in U\).
\end{lemma}
\begin{proof}
  \leanok%
  The proof follows immediately from definitions.
\end{proof}

\begin{lemma}%
  \label{lem:cdh-at-zero}
  \lean{ContDiffHolderAt.zero_exponent_iff}
  \leanok%
  \uses{def:cdh-at}
  A map \(f\colon E\to F\) is \(C^{k+(0)}\) at \(a\) iff it is \(C^{k}\) at \(a\).
\end{lemma}

\begin{proof}
  \leanok%
  The forward implication follows from the definition.
  For the backward implication,
  we need to show that \(D^{k}f(x) - D^{k}f(a)=O(1)\) as \(x\to a\),
  which immediately follows from the continuity of \(D^{k}f\) at \(a\).
\end{proof}

\begin{lemma}%
  \label{lem:cdh-at-of-contDiffAt}
  \uses{def:cdh-at}
  \lean{ContDiffAt.contDiffHolderAt}
  \leanok%
  If \(f\colon E\to F\) is \(C^{l}\) at \(a\) with \(l > k\),
  then it is \(C^{k+(\alpha)}\) at \(a\).
\end{lemma}
\begin{proof}
  \uses{lem:Ck-differentiable-iteratedFDeriv}
  \leanok%
  Since \(D^{k}f\) is differentiable at \(a\), we have \(D^{k}f(x)-D^{k}f(a)=O(x - a)=O(\|x - a\|^{\alpha})\),
  where the latter estimate holds since \(\alpha \le 1\).
\end{proof}

\begin{lemma}%
  \label{lem:cdh-at-mono}
  \uses{def:cdh-at}
  \lean{ContDiffHolderAt.of_exponent_le, ContDiffHolderAt.of_lt, ContDiffHolderAt.of_toLex_le}
  \leanok%
  Let \(f\colon E\to F\) be a map which is \(C^{k+(\alpha)}\) at \(a\).
  Let \(l\) be a natural number and \(\beta \in [0, 1]\) be a real number.
  If \((l, \beta) \le (k, \alpha)\) in the lexicographic order,
  then \(f\) is \(C^{l+(\beta)}\) at \(a\).
\end{lemma}

\begin{proof}
  \uses{lem:cdh-at-of-contDiffAt}
  \leanok%
  Note that \(l\le k\), hence \(f\) is \(C^{l}\) at \(a\).
  In order to show \(D^{l}f(x) - D^{l}f(a) = O(\|x - a\|^{\beta})\) as \(x\to a\),
  consider the cases \(l = k\), \(\beta \le \alpha\) and \(l < k\).

  In the former case, we have \(D^{k}f(x) - D^{k}f(a) = O\left(\|x - a\|^{\alpha}\right)=O\left(\|x - a\|^{\beta}\right)\) as \(x\to a\).

  In the latter case, we apply \autoref{lem:cdh-at-of-contDiffAt}.
\end{proof}

\begin{lemma}%
  \label{lem:cdh-at-prodMk}
  \uses{def:cdh-at}
  If \(f \colon E \to F\) and \(g \colon E \to G\) are \(C^{k+(\alpha)}\) at \(a \in E\),
  then \(x \mapsto (f(x), g(x))\) is \(C^{k+(\alpha)}\) at \(a\).
\end{lemma}
\begin{proof}
  \uses{lem:iteratedFDeriv-prod}
  The lemma immediately follows from \autoref{lem:iteratedFDeriv-prod}.
\end{proof}

\begin{lemma}%
  \label{lem:cdh-at-clm-comp}
  If \(f \colon E \to F\) is \(C^{k+(\alpha)}\) at \(a \in E\)
  and \(g \colon F \to G\) is a continuous linear map,
  then \(g \circ f\) is \(C^{k+(\alpha)}\) at \(a\).
\end{lemma}

\begin{proof}
  Immediately follows from \(D^{k}(g \circ f) = g\circ D^{k}f\).
\end{proof}

\begin{remark}
  \autoref{lem:cdh-at-clm-comp} immediately follows from a more general \autoref{lem:cdh-at-comp} below.
\end{remark}

\begin{lemma}%
  \label{lem:cdh-at-fderiv}
  \uses{def:cdh-at}
  If \(f \colon E \to F\) is \(C^{k+1+(\alpha)}\) at \(a \in E\),
  then \(Df\) is \(C^{k+(\alpha)}\) at \(a\).
\end{lemma}

\begin{corollary}%
  \label{lem:cdh-at-iteratedFDeriv}
  \uses{def:cdh-at}
  If \(f \colon E \to F\) is \(C^{k+l+(\alpha)}\) at \(a \in E\),
  then \(D^{k}f\) is \(C^{l+(\alpha)}\) at \(a\).
\end{corollary}
\begin{proof}
  \uses{lem:cdh-at-fderiv, lem:cdh-at-clm-comp}
  The proof by induction on \(k\) using \autoref{lem:cdh-at-fderiv} and \autoref{lem:cdh-at-clm-comp}.
  The latter lemma is required, because \(D^{k+1}f\) and \(D(D^{k} f)\) have different types in Lean,
  and are related by a continuous linear equivalence.
\end{proof}

\begin{lemma}%
  \label{lem:cdh-at-comp}
  \uses{def:cdh-at}
  Consider \(g\colon F \to G\), \(f\colon E \to F\), and \(a \in E\)
  such that \(g\) is \(C^{k+(\alpha)}\) at \(f(a)\) and \(f\) is \(C^{k+(\alpha)}\) at \(a\), where \(k > 0\) and \(\alpha \in [0, 1]\).
  Then \(g\circ f\) is \(C^{k+(\alpha)}\) at \(a\).
\end{lemma}

\begin{proof}
  Since \(g\) is \(C^{k}\) at \(f(a)\) and \(f\) is \(C^{k}\) at \(a\),
  the composition \(g\circ f\) is \(C^{k}\) at \(a\).
  Let us show that \(D^{k}(g\circ f)(x) - D^{k}(g\circ f)(a) = O(\|x - a\|^{\alpha})\) as \(x\to \alpha\).

  Due to the Faa Di Bruno formula, we have
  \[
    D^k(g\circ f)(a) = \sum_{P}TC_{P}\left(D^{|P|}g(f(a)), D^{|P_{0}|}f(a), \dots, D^{\left|P_{|P|-1}\right|}f(a)\right),
  \]
  where the sum is taken over all partitions \(P\) of \(0, \dots, k - 1\) into disjoint sets,
  and \(TC_{P}\) is a continuous multilinear map in all \(|P|+1\) variables
  which depends only on the normed spaces and the partition \(P\).

  Subtracting these expressions, we see that it suffices to show that for each partition \(P\),
  we have
  \begin{align*}
    TC_{P}&\left(D^{|P|}g(f(x)), D^{|P_{0}|}f(x), \dots, D^{\left|P_{|P|-1}\right|}f(x)\right) -\\
          &TC_{P}\left(D^{|P|}g(f(a)), D^{|P_{0}|}f(a), \dots, D^{\left|P_{|P|-1}\right|}f(a)\right) =\\
          &O(\|x - a\|^{\alpha}).
  \end{align*}
  Since \(k > 0\), the function \(f\) is Lipschitz near \(a\),
  thus \(f(x) - f(a) = O(x - a)\), \(D^{|P|}g(f(x)) - D^{|P|}g(f(a)) = O(\|x - a\|^{\alpha})\), and \(D^{|P_{j}|}f(x) - D^{|P_{j}|}f(a) = O(\|x - a\|^{\alpha})\).
  Now, the goal follows from the fact that a continuous multilinear map is locally Lipschitz.
\end{proof}

\begin{corollary}%
  \label{cor:cdh-at-arith}
  \uses{def:cdh-at}
  Arithmetic operations (addition, subtraction, scalar multiplication, multiplication)
  of \(C^{k+(\alpha)}\) functions is a \(C^{k+(\alpha)}\) function.
\end{corollary}

\begin{proof}
  \uses{lem:cdh-at-comp,lem:cdh-at-of-contDiffAt}
  This fact immediately follows
  from \autoref{lem:cdh-at-comp} and \autoref{lem:cdh-at-of-contDiffAt}.
\end{proof}

\begin{remark}
  It would be nice to formalize the proofs of \autoref{lem:cdh-at-comp} above
  and \autoref{thm:cdh-at-inverse} below
  so that most parts of the proofs
  can be reused in the proofs about a more common class \(C^{k+\alpha}\).
  This can be achieved by writing most of the estimates
  in terms of formal Taylor series satisfying the Faa Di Bruno formula.
\end{remark}

\section{\(C^{k}\) maps with Hölder continuous derivatives on sets}%
\label{sec:chd-on}

\begin{definition}%
  \label{def:cdh-on}
  We say that a map \(f\colon E\to F\) is \emph{\(C^{k+(\alpha)}\) on a pair of sets \(K\), \(U\)}, if
  \begin{itemize}
  \item \(U\) is an open set;
  \item \(K \subset U\);
  \item \(f\) is \(C^{k}\) on \(U\);
  \item for each \(a \in K\), \(D^{k}f(x) - D^{k}f(a) = O(\|x - a\|^{\alpha})\) as \(x\to a\).
  \end{itemize}
\end{definition}

\begin{lemma}%
  \label{lem:exists-cdh-on}
  \uses{def:cdh-at,def:cdh-on}
  Given a map \(f\colon E \to F\), a set \(K\), a natural number \(k\), and a real number \(\alpha \in [0, 1]\),
  there exists \(U\) such that \(f\) is \(C^{k+(\alpha)}\) on \((K, U)\)
  iff \(f\) is \(C^{k+(\alpha)}\) at each \(a \in K\).
\end{lemma}

\begin{lemma}%
  \label{lem:cdh-on-comp}
  \uses{def:cdh-on}
  Suppose that
  \begin{itemize}
  \item \(k > 0\), \(\alpha \in [0, 1]\);
  \item \(g\colon F\to G\) is \(C^{k+(\alpha)}\) on \((t, V)\);
  \item \(f\colon E\to F\) is \(C^{k+(\alpha)}\) on \((s, U)\);
  \item \(f\) maps \(s\) to \(t\);
  \item \(f\) maps \(U\) to \(V\).
  \end{itemize}
  Then \(g\circ f\) is \(C^{k+(\alpha)}\) on \((s, U)\).
\end{lemma}

\begin{proof}
  \uses{lem:cdh-at-comp}
  This statement immediately follows from \autoref{lem:cdh-at-comp}.
\end{proof}

\section{Inverse function theorem}%
\label{sec:inverse-funct-theor}

In this section we prove a version of the inverse function theorem for \(C^{k+(\alpha)}\) functions,
then deduce some corollaries.
For simplicity, we formulate these theorems for finite dimensional real normed spaces,
but most of them should be true in Banach spaces,
possibly with extra assumptions that are automatically satisfied for finite dimensional spaces.

\begin{theorem}%
  \label{thm:cdh-at-inverse}
  \uses{def:cdh-at}
  Consider a map \(f\colon E\to F\) between two finite dimensional real normed spaces.
  Suppose that \(f\) is \(C^{k+(\alpha)}\) at \(a \in E\), \(k > 0\), and \(df(a)\) is bijective.
  Then the inverse function \(f^{-1}\) is \(C^{k+(\alpha)}\) at \(f(a)\).
\end{theorem}
\begin{proof}
  From the usual inverse function theorem for \(C^{k}\) maps,
  we know that \(g=f^{-1}\) is \(C^{k}\) in a neighborhood of \(f(a)\).
  Thus we only need to show that \(D^{k}g(y)-D^{k}g(f(a)) = O\left(\|y - f(a)\|^{\alpha}\right)\) as \(y \to f(a)\).
  Equivalently, we need to show that \(D^{k}g(f(x)) - D^{k}g(f(a)) = O\left(\|x - a\|^{\alpha}\right)\) as \(x\to a\).

  Consider the cases \(k = 1\) and \(k > 1\).

  \textbf{Case I:\@\(k = 1\)}. In this case we need to show that
  \(g'(f(x)) - g'(f(a)) = O\left(\|x - a\|^{\alpha}\right)\) as \(x\to a\).
  This immediately follows from \(g'(f(x)) = {f'(x)}^{-1}\),
  a similar estimate on \(f'(x) - f'(a)\),
  and differentiability of the inversion of a continuous linear equivalence.

  \textbf{Case II:\@\(k > 1\)}.
  Apply the Faa Di Bruno formula to find \(D^{k}(g \circ f) = D^{k} id =0\).
  Note that the only term in the RHS of the Faa Di Bruno formula
  that depends on \(D^{k}g\) is \(D^{k}g\circ_{k} f'\),
  where \(\circ_{k}\) means composition of a continuous multilinear map \(D^{k}g\)
  and a continuous linear map \(f'\).
  Similarly, the only term that depends on \(D^{k}f(x)\) is \(g'(f(x))\circ D^{k}f(x)\).
  All other derivatives of \(f\) and \(g\) are differentiable,
  hence Lipschitz continuous near \(a\) and \(f(a)\), respectively.
  Therefore, \(D^{k}g(f(x))\circ_{k}f'(x) - D^{k}g(f(a))\circ_{k}f'(a)=O\left(\|x - a\|^{\alpha}\right)\) as \(x\to a\).
  This immediately implies the required estimate.
\end{proof}

\begin{lemma}%
  \label{thm:cdh-at-implicit-ker}
  \uses{def:cdh-at}
  Let \(E\), \(F\), and \(G\) be finite dimensional real normed spaces.
  Consider a set \(s \subseteq E \times F\), a point \((a, b) \in s\), and a function \(f\colon E \times F \to G\) such that
  \begin{itemize}
  \item \(f\) is \(C^{k+(\alpha)}\) at every point of \(s\);
  \item \(\left.f\right|_{s}=0\);
  \item the derivative \(D_{2}f\) of \(y \mapsto f(a, y)\) at \(b\) is surjective.
  \end{itemize}
  Let \(S \subseteq F\) be the kernel of \(D_{2}f\).
  Then there exists a map \(\psi\colon E \times S \to F\) and a set \(t \subseteq E \times S\) such that
  \begin{itemize}
  \item \(\psi\) is \(C^{k+(\alpha)}\) at every point of \(t\);
  \item the derivative of \(\psi\) at \((a, 0)\) is the projection to the second coordinate;
  \item the image of a small neighborhood of \((a, 0)\) within \(t\) under the map \((x, y)\mapsto (x, \psi(x, y))\)
    is a small neighborhood of \((a, b)\) within \(s\).
  \end{itemize}
\end{lemma}

\begin{remark}
  We only need this lemma for \(G=\mathbb R\).
  If the proof in this case is a bit easier, then it's OK to formalize this case only.
  Also, there are some versions of the implicit function theorem in Mathlib.
  Possibly, using one of them will be easier than using the inversion function theorem.
\end{remark}

\begin{proof}
  \uses{thm:cdh-at-inverse}
  Let \(\pi\colon F\to S\) be a continuous linear projection, \(\pi \circ \pi = \pi\).
  Consider the map \(h\colon E \times F \to E \times S \times G\)
  given by \(h(x, y) = (x, \pi(y), f(x, y))\).
  Note that \(Dh(a, b) = \id \times \pi \times Df\) is an equivalence,
  hence \(h\) is invertible near \((a, b)\).
  Due to \autoref{thm:cdh-at-inverse}, the inverse function is \(C^{k+(\alpha)}\) at every point of \(h(s)\).
  Also note that \(h\) maps \(s\) to a subset of the subspace \(E\times S\times \{0\}\).
  Let \(\psi(x, z)\) be the second coordinate of \(h^{-1}(x, z, 0)\).
  It is easy to see that this function satisfies all the conditions.
\end{proof}

\begin{theorem}%
  \label{thm:cdh-at-implicit-dichotomy}
  Let \(E\) and \(F\) be finite dimensional real normed spaces.
  Consider a set \(s \subset E \times F\) and \((a, b) \in s\).
  Then
  \begin{itemize}
  \item either \(D(y \mapsto f(a, y))(b) = 0\) for any \(f\colon E \times F\to \mathbb R\)
    such that \(f\) is \(C^{k+(\alpha)}\) at every point of \(s\) and \(\left.f\right|_{s}=0\),
  \item or there exists a hyperplane \(S \subset F\), a set \(t \subset S\), and a function \(\psi\colon E \times S \to F\) such that
    \begin{itemize}
    \item \(\psi\) is \(C^{k+(\alpha)}\) at every point of \(t\);
    \item the derivative of \(\psi\) at \((a, b)\) is the projection to the second coordinate;
    \item the image of a neighborhood of \((a, 0)\) within \(t\) under \((x, y) \mapsto (x, \psi(x, y))\)
      is a neighborhood of \((a, b)\) within \(s\).
    \end{itemize}
  \end{itemize}
\end{theorem}

\chapter{Local estimates}%
\label{cha:local-estimates}

\section{Moreira charts}%
\label{sec:moreira-chart}

The following definition does not appear in the original paper.
However, it helps reformulate Theorem 2.1 from the paper
in a manner that separates estimates from the construction.

While it's possible to generalize most of the material in this section
to a product of any two finite dimensional spaces,
we assume the second space to be the Euclidean space \(\mathbb R^{n}\)
to make definitions and arguments simpler.

\begin{definition}%
  \label{def:moreira-chart}
  \uses{def:cdh-on}
  Consider a natural number \(k\), a real number \(\alpha\in[0, 1]\),
  a set \(s \subset E \times \mathbb R^{n}\), and its open neighborhood \(U\).
  A \emph{Moreira chart} of depth \(l \le k\) consists of
  \begin{itemize}
  \item an antitone sequence of dimensions \(n = d_{0}\ge d_{1}\ge \dots \ge d_{l}\);
  \item for each \(i \le l\), an open ball \(B_{i} \subset E\) and an open ball \(V_{i}\subset \mathbb R^{d_{i}}\);
  \item for each \(i \le l\), a set \(t_{i}\subset B_{i}\times V_{i}\);
  \item for each \(i < l\), a map \(\psi_{i}\colon E\times \mathbb R^{d_{i+1}} \to \mathbb R^{d_{i}}\)
  \end{itemize}
  such that
  \begin{itemize}
  \item \(B_{0}\times V_{0}\subset U\) and \(t_{0}\subset s\);
  \item each map \((x, y) \mapsto (x, \psi_{i}(x, y))\)
    maps \(B_{i+1}\times V_{i+1}\) to \(B_{i}\times V_{i}\) and \(t_{i+1}\) to \(t_{i}\);
  \item each \(\psi_{i}\) is \(C^{k-i+(\alpha)}\) on \((t_{i+1}, B_{i+1}\times V_{i+1})\);
  \item each \(\psi_{i}(x, y)\) is expanding in \(y\) on \(V_{i+1}\) for each \(x \in B_{i + 1}\);
  \item for any map \(f\colon E\times \mathbb R^{d_{i}} \to \mathbb R\)
    that is \(C^{k-i+(\alpha)}\) on \((t_{i}, B_{i}\times V_{i})\) and vanishes on \(t_{i}\),
    the differential \(D(f \circ \psi_{i})\) vanishes on \(t_{i+1}\).
  \end{itemize}
\end{definition}

\begin{definition}%
  \label{def:moreira-chart-map}
  \uses{def:moreira-chart}
  Given a Moreira chart, for each \(i \le l\) we define the map \(\Psi_{i}\colon E\times \mathbb R^{d_{i}}\to \mathbb R^{n}\)
  by \(\Psi_{0}(x, y)=y\) and \(\Psi_{i+1}(x, y)=\Psi_{i}(x, \psi_{i}(x, y))\).
\end{definition}

\begin{lemma}%
  \label{lem:moreira-chart-map-props}
  \uses{def:moreira-chart-map}
  Given a \(C^{k+(\alpha)}\) Moreira chart,
  \begin{itemize}
  \item each map \(\Psi_{i}\) is \(C^{k-i+(\alpha)}\) on \((t_{i+1}, B_{i+1}\times V_{i+1})\);
  \item \((x, \Psi_{i}(x, y)) \in s\) whenever \((x, y) \in t_{i+1}\);
  \item each \(\Psi_{i}(x, y)\) is expanding in \(y\) on \(V_{i+1}\) for each \(x\in B_{i+1}\).
  \end{itemize}
\end{lemma}

\section{Estimates on \(C^{1+(\alpha)}\) functions}

\begin{lemma}%
  \label{lem:cdh-at-sub-affine-le-of-meas}
  Let \(E\) and \(F\) be a real normed spaces.
  Consider a function \(f\colon E\to F\), points \(a, b \in E\),
  and numbers \(C\ge 0\), \(\delta\ge 0\), \(r \ge 0\) such that
  \begin{itemize}
  \item \(f\) is differentiable on \([a, b]\);
  \item \(\|D_{b - a}f(a + t(b - a))\| \le Ct^{r}{\|b - a\|}^{1+r}\);
  \item the set of \(t\) such that \(D_{b - a}f(a + t(b - a)) = 0\)
    has measure at least \(1 - \delta\).
  \end{itemize}
  Then \(\|f(b) - f(a)\| \le C\delta{\|b - a\|}^{1+r}\).
\end{lemma}

\begin{proof}
  Let \(s\) be the set of \(t \in [0, 1]\) such that \(D_{b - a}f(a + t(b - a)) \ne 0\).
  We have
  \begin{align*}
    \|f(b) - f(a)\| &= \left|\int_{0}^{1} D_{b - a}f(a + t(b - a))\,dt\right| \\
                    &= \left|\int_{t\in s} D_{b - a}f(a + t(b - a))\,dt\right| \\
                    &\le \int_{t \in s} Ct^{r}{\|b - a\|}^{1+r}\,dt \\
                    &= C{\|b - a\|}^{1+r} \int_{t \in s}t^{r}\,dt \\
                    &\le C{\|b - a\|}^{1+r} \lambda(s) \\
                    &\le C\delta {\|b - a\|}^{1+r}
  \end{align*}
\end{proof}

\begin{remark}
  The constant can be improved, but probably we don't need this.
  In order to do this, we need to show that
  \[
    \int_{t \in s}t^{r}\,dt \le \int_{1 - \delta}^{1} t^{r}\,dt = \frac{1 - {(1 - \delta)}^{r + 1}}{r + 1}.
  \]
\end{remark}

\begin{corollary}%
  \label{cor:cdh-at-sub-affine-le-of-meas-fderiv}
  Let \(E\) be a finite dimensional real normed space.
  Let \(F\) be a real normed space.
  Consider a function \(f\colon E\to F\), points \(a, b \in E\),
  and numbers \(C\ge 0\), \(\delta\ge 0\), \(r \ge 0\) such that
  \begin{itemize}
  \item \(f\) is differentiable on \([a, b]\);
  \item \(\|Df(a + t(b - a))\| \le Ct^{r}{\|b - a\|}^r\);
  \item the set of \(t\) such that \(D_{b - a}f(a + t(b - a)) = 0\)
    has measure at least \(1 - \delta\).
  \end{itemize}
  Then \(\|f(b) - f(a)\| \le C\delta{\|b - a\|}^{1+r}\).
\end{corollary}

\begin{proof}
  \uses{lem:cdh-at-sub-affine-le-of-meas}
  This lemma immediately follows from \autoref{lem:cdh-at-sub-affine-le-of-meas}
  and the fact that \(\|D_{b-a}f(x)\|\le \|Df(x)\|\times\|b - a\|\).
\end{proof}

\begin{corollary}%
  \label{cor:sub-isBigO-rpow-of-fderiv}
  
\end{corollary}

\begin{lemma}%
  \label{lem:cdh-at-sub-affine-isBigO}
  \uses{def:cdh-at}
  If \(f\colon E \to F\) is \(C^{1+(\alpha)}\) at \(a\),
  then \(f(x) - f(a) - Df(a)(x - a) = O\left(\|x - a\|^{1 + \alpha}\right)\) as \(x \to a\).
\end{lemma}
Here \(Df(a)(x - a)\) means differential of \(f\) at \(a\) applied to \(x - a\).

\begin{proof}
  We have
  \begin{align*}
    f(x) - f(a) - Df(a)(x - a) &= \int_{0}^{1}\left(Df(a + t(x - a)) - Df(a)\right)(x - a)\,dt\\
                               &= O\left(\int_{0}^{1}t^{\alpha}\|x - a\|^{1+\alpha}\,dt\right)\\
                               &= O\left(\|x - a\|^{1 + \alpha}\right).
  \end{align*}
\end{proof}

\begin{remark}
  An alternative proof of \autoref{lem:cdh-at-sub-affine-isBigO}
  applies \autoref{lem:cdh-at-sub-affine-le-of-meas} to \(\delta=1\).
\end{remark}

\begin{corollary}%
  \label{cor:cdh-at-sub-isBigO}
  \uses{def:cdh-at}
  If \(f\colon E \to F\) is \(C^{1+(\alpha)}\) at \(a\) and \(Df(a) = 0\),
  then \(f(x) - f(a) = O\left(\|x - a\|^{1 + \alpha}\right)\) as \(x \to a\).
\end{corollary}

\begin{proof}
  \uses{lem:cdh-at-sub-affine-isBigO}
  Immediately follows from \autoref{lem:cdh-at-sub-affine-isBigO} and \(Df(a) = 0\).
\end{proof}


%%% Local Variables:
%%% mode: latex
%%% TeX-master: "print.tex"
%%% End:
